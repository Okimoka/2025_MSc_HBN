% !TeX spellcheck = en-US
% LTeX: language=en-US
% !TeX encoding = utf8
% !TeX program = lualatex
% !TeX TXS-program:compile = txs:///lualatex/[--shell-escape]
% !BIB program = biber
% -*- coding:utf-8 mod:LaTeX -*-

% The following package allows \\ at the title page
% For more information see https://github.com/latextemplates/scientific-thesis-cover/issues/4
\RequirePackage{kvoptions-patch}

\documentclass[
  % fontsize=11pt is the standard
  numbers=noenddot,
  english,  % English as main language; this parameter is passed to other packages (e.g., selnolig in the case of lualatex)
  a4paper,  % KOMAScript allows for both paper=a4 and (standard) a4paper - https://tex.stackexchange.com/a/61044/9075
  twoside,  % We are optimizing for both screen and two-side printing. So the page numbers will jump, but the content is configured to stay in the middle (by using the geometry package)
  bibliography=totoc,
  % idxtotoc,   % Index ins Inhaltsverzeichnis
  % liststotoc, % List of * ins Inhaltsverzeichnis, mit liststotocnumbered werden die Abbildungsverzeichnisse nummeriert
  headsepline,
  cleardoublepage=empty,
  parskip=half,
  %               draft    % um zu sehen, wo noch nachgebessert werden muss - wichtig, da Bindungskorrektur mit drin
  draft=false
]{scrbook}
\usepackage{scrlayer-scrpage}

\usepackage{iftex}

\usepackage{ifplatform}

% backticks (`) are rendered as such in verbatim environments.
% See following links for details:
%   - https://tex.stackexchange.com/a/341057/9075
%   - https://tex.stackexchange.com/a/47451/9075
%   - https://tex.stackexchange.com/a/166791/9075
\usepackage{upquote}
% In preamble:
\newif\ifdraft
\drafttrue        % comment out for “final”
% \draftfalse

\ifdraft
  \newcommand{\mynote}[1]{\textcolor{red}{[#1]}}
\else
  \newcommand{\mynote}[1]{}
\fi

\usepackage{xcolor}
% Set English as language and allow to write hyphenated"=words
%
% Even though `american`, `english` and `USenglish` are synonyms for babel package (according to https://tex.stackexchange.com/questions/12775/babel-english-american-usenglish), the llncs document class is prepared to avoid the overriding of certain names (such as "Abstract." -> "Abstract" or "Fig." -> "Figure") when using `english`, but not when using the other 2.
% english has to go last to set it as default language
\usepackage[ngerman,main=english]{babel}
%
% Hint by http://tex.stackexchange.com/a/321066/9075 -> enable "= as dashes
\addto\extrasenglish{\languageshorthands{ngerman}\useshorthands{"}}

% Links behave as they should. Enables "\url{...}" for URL typesettings.
% Allow URL breaks also at a hyphen, even though it might be confusing: Is the "-" part of the address or just a hyphen?
% See https://tex.stackexchange.com/a/3034/9075.
\usepackage[hyphens]{url}

% When activated, use text font as url font, not the monospaced one.
% For all options see https://tex.stackexchange.com/a/261435/9075.
% \urlstyle{same}

% Improve wrapping of URLs - hint by http://tex.stackexchange.com/a/10419/9075
\makeatletter
\g@addto@macro{\UrlBreaks}{\UrlOrds}
\makeatother

% nicer // - solution by http://tex.stackexchange.com/a/98470/9075
% DO NOT ACTIVATE -> prevents line breaks
%\makeatletter
%\def\Url@twoslashes{\mathchar`\/\@ifnextchar/{\kern-.2em}{}}
%\g@addto@macro\UrlSpecials{\do\/{\Url@twoslashes}}
%\makeatother

%math stuff
\usepackage[
  centertags,    % (default) center tags vertically
  % tbtags,        % 'Top-or-bottom tags': For a split equation, place equation numbers level with the last (resp. first) line, if numbers are on the right (resp. left).
  sumlimits,    % (default) Place the subscripts and superscripts of summation symbols above and below
  % nosumlimits,   % Always place the subscripts and superscripts of summation-type symbols to the side, even in displayed equations.
  intlimits,     % Like sumlimits, but for integral symbols.
  % nointlimits,   % (default) Opposite of intlimits.
  namelimits,    % (default) Like sumlimits, but for certain 'operator names' such as det, inf, lim, max, min, that traditionally have subscripts placed underneath when they occur in a displayed equation.
  % nonamelimits,  % Opposite of namelimits.
  % leqno,         % Place equation numbers on the left.
  % reqno,         % Place equation numbers on the right.
  fleqn,         % Position equations at a fixed indent from the left margin rather than centered in the text column.
]{amsmath}
\SetMathAlphabet{\mathcal}{normal}{OMS}{amsa}{m}{n} %% AMS font for mathcal

%%% Doc: http://mirror.ctan.org/tex-archive/macros/latex/contrib/mh/doc/mathtools.pdf
% Erweitert amsmath und behebt einige Bugs
\usepackage[fixamsmath,disallowspaces]{mathtools}

%%% Doc: http://www.ctan.org/info?id=fixmath
% LaTeX's default style of typesetting mathematics does not comply
% with the International Standards ISO31-0:1992 to ISO31-13:1992
% which indicate that uppercase Greek letters always be typeset
% upright, as opposed to italic (even though they usually
% represent variables) and allow for typesetting of variables in a
% boldface italic style (even though the required fonts are
% available). This package ensures that uppercase Greek be typeset
% in italic style, that upright $\Delta$ and $\Omega$ symbols are
% available through the commands \upDelta and \upOmega; and
% provides a new math alphabet \mathbold for boldface
% italic letters, including Greek.
\usepackage{fixmath}

%for theorems, replacement for amsthm
\usepackage[amsmath,hyperref]{ntheorem}
\theorempreskipamount 2ex plus1ex minus0.5ex
\theorempostskipamount 2ex plus1ex minus0.5ex
\theoremstyle{break}
\newtheorem{definition}{Definition}[chapter]

%%% Doc: http://mirror.ctan.org/tex-archive/macros/latex/contrib/onlyamsmath/onlyamsmath.dvi
% Warnt bei Benutzung von Befehlen die mit amsmath inkompatibel sind.

% Braucht man evtl. nicht.
% \usepackage[
% 	all,
% 	warning
% ]{onlyamsmath}

%% !!! If you change the font, be sure that words such as "workflow" can
%% !!! still be copied from the PDF. If this is not the case, you have
%% !!! to use glyphtounicode. See comment at cmap package.
%%
%% Background: "workflow" contains "fl" which is a ligature, which in turn
%%             is rendered as one character in the PDF and needs to be split
%%             whily copying.

\ifluatex
  \usepackage[no-math]{fontspec}
  \usepackage{unicode-math}

  % See https://tug.org/FontCatalogue/texgyretermes/ for more information
  \setmainfont{texgyretermes}[
    Extension = .otf,
    UprightFont = *-regular,
    BoldFont = *-bold,
    ItalicFont = *-italic,
    BoldItalicFont = *-bolditalic,
    Ligatures=TeX
  ]
  % See https://tug.org/FontCatalogue/texgyreheros/ for more information
  \setsansfont[Scale=.9]{TeX Gyre Heros Regular}
  % shapely l, upright quotes
  % Normal scaling is too large --> thus, we use ",Scale=.9"
  \ifwindows
    \setmonofont[StylisticSet={1,3},Scale=.9]{Inconsolata}
  \else
    \setmonofont[StylisticSet={1,3},Scale=.9]{Inconsolatazi4}
  \fi

  % Enable proper ligatures
  % For more information see https://ctan.org/pkg/selnolig
  % language "english" or "ngerman" is passed to selnolig by the document class
  \usepackage{selnolig}

\else
  \RequirePackage{newtxtext}
  \RequirePackage{newtxmath}
  \RequirePackage[zerostyle=b,scaled=.9]{newtxtt}

  % Has to be loaded AFTER any font packages. See https://tex.stackexchange.com/a/2869/9075.
  \usepackage[T1]{fontenc}
\fi

% DE: Noch mehr Symbole
%\usepackage{stmaryrd} %fuer \ovee, \owedge, \otimes
%\usepackage{marvosym} %fuer \Writinghand %patched to not redefine \Rightarrow
%\usepackage{mathrsfs} %mittels \mathscr{} schoenen geschwungenen Buchstaben erzeugen
%\usepackage{calrsfs} %\mathcal{} ein bisserl dickeren buchstaben erzeugen - sieht net so gut aus.

% EN: Fallback font - if the subsequent font packages do not define a font (e.g., monospaced)
%     This is the modern package for "Computer Modern".
%     In case this gets activated, one has to switch from cmap package to glyphtounicode (in the case of pdflatex)
% DE: Fallback-Schriftart
%\usepackage[%
%    rm={oldstyle=false,proportional=true},%
%    sf={oldstyle=false,proportional=true},%
%    tt={oldstyle=false,proportional=true,variable=true},%
%    qt=false%
%]{cfr-lm}

% EN: Headings are typeset in Helvetica (which is similar to Arial)
% DE: Schriftart fuer die Ueberschriften - ueberschreibt lmodern
%\usepackage[scaled=.95]{helvet}

% DE: Für Schreibschrift würde tun, muss aber nicht
%\usepackage{mathrsfs} %  \mathscr{ABC}

% EN: Font for the main text
% DE: Schriftart fuer den Fliesstext - ueberschreibt lmodern
%     Linux Libertine, siehe http://www.linuxlibertine.org/
%     Packageparamter [osf] = Minuskel-Ziffern
%     rm = libertine im Brottext, Linux Biolinum NICHT als serifenlose Schrift, sondern helvet (von oben) beibehalten
%\usepackage[rm]{libertine}

% EN: Alternative Font: Palantino. It is recommeded by Prof. Ludewig for German texts
% DE: Alternative Schriftart: Palantino, Packageparamter [osf] = Minuskel-Ziffern
%     Bitte nur in deutschen Texten
%\usepackage{mathpazo} %ftp://ftp.dante.de/tex-archive/fonts/mathpazo/ - Tipp aus DE-TEX-FAQ 8.2.1
% EN: The euro sign
% DE: Das Euro Zeichen
%     Fuer Palatino (mathpazo.sty): richtiges Euro-Zeichen
%     Alternative: \usepackage{eurosym}
% \newcommand{\EUR}{\ppleuro}

% DE: Schriftart fuer Programmcode - ueberschreibt lmodern
%     Falls auskommentiert, wird die Standardschriftart lmodern genommen
%     Fuer schreibmaschinenartige Schluesselwoerter in den Listings - geht bei alten Installationen nicht, da einige Fontshapes (<>=) fehlen
%\usepackage[scaled=.92]{luximono}
%\usepackage{courier}
% DE: BeraMono als Typewriter-Schrift, Tipp von http://tex.stackexchange.com/a/71346/9075
%\usepackage[scaled=0.83]{beramono}

\usepackage{setspace}
% Alternative package: https://ctan.org/pkg/leading

% Symbole Check und Cross
\usepackage{pifont}
\newcommand{\dingcheck}{\ding{51}}
\newcommand{\dingcross}{\ding{55}}
%for scaling see http://tex.stackexchange.com/a/130236/9075

% DE: Noch mehr Symbole
%\usepackage{stmaryrd} %fuer \ovee, \owedge, \otimes
%\usepackage{marvosym} %fuer \Writinghand %patched to not redefine \Rightarrow
%\usepackage{mathrsfs} %mittels \mathscr{} schoenen geschwungenen Buchstaben erzeugen
%\usepackage{calrsfs} %\mathcal{} ein bisserl dickeren buchstaben erzeugen - sieht net so gut aus.

\automark[section]{chapter}
\setkomafont{pageheadfoot}{\normalfont\sffamily}
\setkomafont{pagenumber}{\normalfont\sffamily}

\ihead[]{}
\chead[]{}
\ohead[]{\headmark}
\cfoot[]{}
\ofoot[\usekomafont{pagenumber}\thepage]{\usekomafont{pagenumber}\thepage}
\ifoot[]{}

% Character protrusion and font expansion. See http://www.ctan.org/tex-archive/macros/latex/contrib/microtype/

\usepackage[
  babel=true, % Enable language-specific kerning. Take language-settings from the languge of the current document (see Section 6 of microtype.pdf)
  expansion=alltext,
  protrusion=alltext-nott, % Ensure that at listings, there is no change at the margin of the listing
  % In the standard configuration, this template is always in the final mode, so this option only makes a difference if "pros" use the draft mode
  final % Always enable microtype, even if in draft mode. This helps finding bad boxes quickly.
]{microtype}

% \texttt{test -- test} keeps the "--" as "--" (and does not convert it to an en dash)
\DisableLigatures{encoding = T1, family = tt* }

%\DeclareMicrotypeSet*[tracking]{my}{ font = */*/*/sc/* }%
%\SetTracking{ encoding = *, shape = sc }{ 45 }
% Source: http://homepage.ruhr-uni-bochum.de/Georg.Verweyen/pakete.html
% Deactiviated, because does not look good

\usepackage{graphicx}

% Base folder, so there is no need to repeat this over and over again.
\graphicspath{ {figures/} }

%%% Doc: http://mirror.ctan.org/tex-archive/macros/latex/contrib/pdfpages/pdfpages.pdf
\usepackage{pdfpages} % Include pages from external PDF documents in LaTeX documents

% Diagonal lines in a table - http://tex.stackexchange.com/questions/17745/diagonal-lines-in-table-cell
% Slashbox is not available in texlive (due to licensing) and also gives bad results. Thus, we use diagbox
\usepackage{diagbox}

\ifluatex
  \usepackage{spelling}
  \spellingoutput{off}
\fi

\usepackage[dvipsnames, table]{xcolor}
% Code Listings
\usepackage{listings}

\definecolor{eclipseStrings}{RGB}{42,0.0,255}
\definecolor{eclipseKeywords}{RGB}{127,0,85}
\colorlet{numb}{magenta!60!black}

% JSON definition
% Source: https://tex.stackexchange.com/a/433961/9075

\lstdefinelanguage{json}{
  basicstyle=\normalfont\ttfamily,
  commentstyle=\color{eclipseStrings}, % style of comment
  stringstyle=\color{eclipseKeywords}, % style of strings
  numbers=left,
  numberstyle=\scriptsize,
  stepnumber=1,
  numbersep=8pt,
  showstringspaces=false,
  breaklines=true,
  frame=lines,
  % backgroundcolor=\color{gray}, %only if you like
  string=[s]{"}{"},
  comment=[l]{:\ "},
  morecomment=[l]{:"},
  literate=
    *{0}{{{\color{numb}0}}}{1}
    {1}{{{\color{numb}1}}}{1}
    {2}{{{\color{numb}2}}}{1}
    {3}{{{\color{numb}3}}}{1}
    {4}{{{\color{numb}4}}}{1}
    {5}{{{\color{numb}5}}}{1}
    {6}{{{\color{numb}6}}}{1}
    {7}{{{\color{numb}7}}}{1}
    {8}{{{\color{numb}8}}}{1}
    {9}{{{\color{numb}9}}}{1}
}

\lstset{
  % everything between (* *) is a latex command
  escapeinside={(*}{*)},
  %
  language=json,
  %
  showstringspaces=false,
  %
  extendedchars=true,
  %
  basicstyle=\footnotesize\ttfamily,
  %
  commentstyle=\slshape,
  %
  % default: \rmfamily
  stringstyle=\ttfamily,
  %
  breaklines=true,            % Zeilen werden umbrochen
  %
  breakatwhitespace=true,
  %
  % alternative: fixed
  columns=flexible,
  %
  tabsize=2,                  % Groesse von Tabs
  %
  numbers=left,
  %
  numberstyle=\tiny,
  %
  basewidth=.5em,
  %
  xleftmargin=.5cm,
  %
  % aboveskip=0mm,
  %
  % belowskip=0mm,
  %
  captionpos=b
}
\ifpdftex

  % Enable Umlauts when using \lstinputputlisting.
  % See https://stackoverflow.com/a/29260603/873282 für details.
  % listingsutf8 did not work in June 2020.
  \lstset{literate=
    {á}{{\'a}}1 {é}{{\'e}}1 {í}{{\'i}}1 {ó}{{\'o}}1 {ú}{{\'u}}1
  {Á}{{\'A}}1 {É}{{\'E}}1 {Í}{{\'I}}1 {Ó}{{\'O}}1 {Ú}{{\'U}}1
  {à}{{\`a}}1 {è}{{\`e}}1 {ì}{{\`i}}1 {ò}{{\`o}}1 {ù}{{\`u}}1
  {À}{{\`A}}1 {È}{{\'E}}1 {Ì}{{\`I}}1 {Ò}{{\`O}}1 {Ù}{{\`U}}1
  {ä}{{\"a}}1 {ë}{{\"e}}1 {ï}{{\"i}}1 {ö}{{\"o}}1 {ü}{{\"u}}1
  {Ä}{{\"A}}1 {Ë}{{\"E}}1 {Ï}{{\"I}}1 {Ö}{{\"O}}1 {Ü}{{\"U}}1
  {â}{{\^a}}1 {ê}{{\^e}}1 {î}{{\^i}}1 {ô}{{\^o}}1 {û}{{\^u}}1
  {Â}{{\^A}}1 {Ê}{{\^E}}1 {Î}{{\^I}}1 {Ô}{{\^O}}1 {Û}{{\^U}}1
  {Ã}{{\~A}}1 {ã}{{\~a}}1 {Õ}{{\~O}}1 {õ}{{\~o}}1
  {œ}{{\oe}}1 {Œ}{{\OE}}1 {æ}{{\ae}}1 {Æ}{{\AE}}1 {ß}{{\ss}}1
  {ű}{{\H{u}}}1 {Ű}{{\H{U}}}1 {ő}{{\H{o}}}1 {Ő}{{\H{O}}}1
  {ç}{{\c c}}1 {Ç}{{\c C}}1 {ø}{{\o}}1 {å}{{\r a}}1 {Å}{{\r A}}1
  }
\fi

\lstloadlanguages{% Check dokumentation for further languages...
  %[Visual]Basic
  %Pascal
  %C
  %C++
  %XML
  %HTML
}

% For easy quotations: \enquote{text}
% This package is very smart when nesting is applied, otherwise textcmds (see below) provides a shorter command
\usepackage[autostyle=true]{csquotes}

% Enable using "`quote"' - see https://tex.stackexchange.com/a/150954/9075
\defineshorthand{"`}{\openautoquote}
\defineshorthand{"'}{\closeautoquote}

% Nicer tables (\toprule, \midrule, \bottomrule)
\usepackage{booktabs}

% Extended enumerate, such as \begin{compactenum}
\usepackage{paralist}

\usepackage[
  backend=biber,
  style=numeric-comp,
  sorting=none,
  autocite=superscript
]{biblatex}

%\usepackage[
%  backend       = biber, %biber does not work with 64x versions alternative: bibtex8; %minalphanames only works with biber backend
%  sortcites     = true,
%  bibstyle      = alphabetic
%  citestyle     = alphabetic,
%  giveninits    = true,
%  useprefix     = false, %"von, van, etc." will be printed, too. See below.
%  minnames      = 1,
%  minalphanames = 3,
%  maxalphanames = 4,
%  maxbibnames   = 99,
%  maxcitenames  = 2,
%  natbib        = true,
%  eprint        = true,
%  url           = true,
%  doi           = true, %source: http://tex.stackexchange.com/a/23118/9075
%  isbn          = true, %source: http://tex.stackexchange.com/a/23118/9075
%  backref       = true]{biblatex}

% enable more breaks at URLs. See https://tex.stackexchange.com/a/134281.
\setcounter{biburllcpenalty}{7000}
\setcounter{biburlucpenalty}{8000}

\bibliography{bibliography}
%\addbibresource[datatype=bibtex]{\bibliography{bibliography}}

% Do not put "vd" in the label, but put it at "\citeauthor"
% Source: http://tex.stackexchange.com/a/30277/9075
\makeatletter
\AtBeginDocument{\toggletrue{blx@useprefix}}
\AtBeginBibliography{\togglefalse{blx@useprefix}}
\makeatother

% Thin spaces between initials
% http://tex.stackexchange.com/a/11083/9075
\renewrobustcmd*{\bibinitdelim}{\,}

% Keep first and last name together in the bibliography
% http://tex.stackexchange.com/a/196192/9075
\renewcommand*\bibnamedelimc{\addnbspace}
\renewcommand*\bibnamedelimd{\addnbspace}

% Replace last "and" by comma in bibliography
% See http://tex.stackexchange.com/a/41532/9075
\AtBeginBibliography{%
  \renewcommand*{\finalnamedelim}{\addcomma\space}%
}

% enable hyperlinked author names when using \citeauthor
% source: http://tex.stackexchange.com/a/75916/9075
\DeclareCiteCommand{\citeauthor}
{
  \boolfalse{citetracker}%
  \boolfalse{pagetracker}%
  \usebibmacro{prenote}
}
{
  \ifciteindex
  {\indexnames{labelname}}
  {}%
  \printtext[bibhyperref]{\printnames{labelname}}
}
{\multicitedelim}
{\usebibmacro{postnote}}

% Farbige Tabellen
% ----------------
% Das Paket colortbl wird inzwischen automatisch durch xcolor geladen
%
% Erweiterte Funktionen innerhalb von Tabellen
% --------------------------------------------
%%% Doc: http://mirror.ctan.org/tex-archive/macros/latex/contrib/multirow/multirow.sty
\usepackage{multirow} % Mehrfachspalten
%
%%% Doc: Documentation inside dtx Package
\usepackage{dcolumn}  % Ausrichtung an Komma oder Punkt

%%% Doc: http://mirror.ctan.org/tex-archive/macros/latex/contrib/supertabular/supertabular.pdf
%\usepackage{supertabular}

%%% Fussnoten/Endnoten ===================================================

%%% Doc: http://mirror.ctan.org/tex-archive/macros/latex/contrib/footmisc/footmisc.pdf
%
\usepackage[
  bottom,      % Footnotes appear always on bottom. This is necessary specially when floats are used
  stable,      % Make footnotes stable in section titles
  % perpage,     % Reset on each page
  % para,        % Place footnotes side by side of in one paragraph.
  % side,        % Place footnotes in the margin
  ragged,      % Use RaggedRight
  % norule,      % Suppress rule above footnotes
  multiple,    % Rearrange multiple footnotes intelligent in the text.
  % symbol,      % Use symbols instead of numbers
]{footmisc}

\counterwithout{footnote}{chapter} % Continuous numbering of footnotes across chapters

\interfootnotelinepenalty=10000 % Verhindert das Fortsetzen von Fussnoten auf der gegenüberligenden Seite

% EN: Put footnotes below floats
% DE: Fußnoten unter Gleitumgebungen ("floats") platzieren
% Source: https://tex.stackexchange.com/a/32993/9075
\usepackage{stfloats}
\fnbelowfloat

% EN: Extended support for footnotes
% DE: Fußnoten
%
%\usepackage{dblfnote}  %Zweispaltige Fußnoten
%
% Keine hochgestellten Ziffern in der Fußnote (KOMA-Script-spezifisch):
%\deffootnote[1.5em]{0pt}{1em}{\makebox[1.5em][l]{\bfseries\thefootnotemark}}
%
% Abstand zwischen Fußnoten vergrößern:
%\setlength{\footnotesep}{.85\baselineskip}
%
% EN: Following command disables the separting line of the footnote
% DE: Folgendes Kommando deaktiviert die Trennlinie zur Fußnote
%\renewcommand{\footnoterule}{}
%
%\addtolength{\skip\footins}{\baselineskip} % Abstand Text <-> Fußnote

% DE: Fußnoten immer ganz unten auf einer \raggedbottom-Seite
% DE: fnpos kommt aus dem yafoot package
%\usepackage{fnpos}
%\makeFNbelow
%\makeFNbottom

% TODO (and comment) configuration
%
% - \todo (from todo, easy-todo, todonotes) / \TODO (from fixmetodonotes) - for "normal" TODOs
% - \todofix - "important" TODOs
%
% - \textcomment - highlights text and has a hover comment
% - \sidecomment - just puts a comment to the side. Note: \comment MUST NOT be used as command name, it is already defined by much packages (mathdesign, mindflow, verbatim, and others)
%
% - \missingfigure
%
% - \textmarker
% - \modified
% - \change      - adresses a review comment

% Enable nice comments
\usepackage{pdfcomment}

\newcommand{\textcomment}[2]{\colorbox{yellow!60}{#1}\pdfcomment[color={0.234 0.867 0.211},hoffset=-6pt,voffset=10pt,opacity=0.5]{#2}}

% Small PDF comment
% 1. Parameter: Comment
\newcommand{\sidecomment}[1]{\pdfcomment[color={0.045 0.278 0.643},voffset=4pt,icon=Note]{#1}}
% Disabled variant - for the final PDF
%\newcommand{\sidecomment}[1]{}

\newcommand{\todo}[1]{TODO!\sidecomment{#1}}

% Änderungen
%
% 1. Parameter: Review-Kommentar
% 2. Parameter: Neuer Text
\newcommand{\change}[2]{{\color{red}#2}\pdfcomment[color={0.234 0.867 0.211},voffset=8pt,opacity=0.5]{#1}}
% Disabled variant - for the final PDF
%\newcommand{\change}[2]{#2}

% Define default commands
\makeatletter
\@ifundefined{missingfigure}{\newcommand{\missingfigure}{... missing figure ...}}{}
\@ifundefined{textcomment}{\newcommand{\textcomment}[2]{#1 \todo{#2}}}{}
\@ifundefined{sidecomment}{\newcommand{\sidecomment}[1]{\marginpar{#1}}}{}
\@ifundefined{todo}{\newcommand{\todo}[1]{\sidecomment{#1}}}{}
\@ifundefined{TODO}{\newcommand{\TODO}[1]{\todo{#1}}}{}
\@ifundefined{todofix}{\newcommand{\todofix}[1]{\todo{#1}}}{}
\@ifundefined{change}{\newcommand{\change}[2]{#1 $\rightarrow$ #2}}{}
\makeatother

% Textmarker (Textfarbe rot)
\newcommand{\textmarker}[1]{{\color{red} #1}\xspace}

% Modified (Text blau)
\newcommand{\modified}[1]{{\color{blue!60!black} #1}\xspace}

\usepackage[group-minimum-digits=4,per-mode=fraction]{siunitx}

% See http://tex.stackexchange.com/a/83051/9075
% Normally, doesn't work with hyperref, but cleveref fixes that
\usepackage{varioref}

% Enable that parameters of \cref{}, \ref{}, \cite{}, ... are linked so that a reader can click on the number an jump to the target in the document
\usepackage{hyperref}

% Enable hyperref without colors and without bookmarks
\hypersetup{
  hidelinks,
  colorlinks=true,       % Links erhalten Farben statt Kaeten
  raiselinks=true,       % calculate real height of the link
  allcolors=black,
  pdfstartview=Fit,
  breaklinks=true,       % Links ueberstehen Zeilenumbruch
  hypertexnames=false,   % Fix jumping to algorithm line - http://tex.stackexchange.com/a/156404/9075
}

% Enable correct jumping to figures when referencing
\usepackage[all]{hypcap}


%%%
% Ermoeglicht es, Abbildungen um 90 Grad zu drehen
% Alternatives Paket: rotating Allerdings wird hier nur das Bild gedreht, während bei lscape auch die PDF-Seite gedreht wird.
%Das Paket lscape dreht die Seite auch nicht
\usepackage{pdflscape}

\usepackage[caption=false,font=footnotesize]{subfig}

% Alternative for making subfigures:
% Part of the caption package. See http://www.ctan.org/pkg/caption
% Ersetzt die Pakete subfigure und subfig - siehe https://tex.stackexchange.com/a/13778/9075
%
% (subfigure is outdated. subfig is maintained, but subcaption is better)
% See: http://tex.stackexchange.com/questions/13625/subcaption-vs-subfig-best-package-for-referencing-a-subfigure
%\usepackage[hypcap=true]{subcaption}

\usepackage{mindflow}

% https://ctan.org/pkg/algorithms
% Consists of two environments: algorithm and algorithmic
% Although oudated, it defines the "algorithm" float enviornment
% TODO: Define floating environment "algorithm" in other ways
\usepackage[chapter]{algorithm}

% https://ctan.org/pkg/algpseudocodex
% Successor of algorithmicx; more modern than https://ctan.org/pkg/algorithms
\usepackage{algpseudocodex}

\floatname{algorithm}{Algorithmus}
\renewcommand{\listalgorithmname}{Algorithmenverzeichnis}

\newcommand{\commentchar}{\ensuremath{/\mkern-4mu/}}
\algrenewcommand{\algorithmiccomment}[1]{\hfill $\commentchar$ #1}

% Extensions for references inside the document (\cref{fig:sample}, ...)
% Enable usage \cref{...} and \Cref{...} instead of \ref: Type of reference included in the link
% That means, "Figure 5" is a full link instead of just "5".
\usepackage[capitalise,nameinlink,noabbrev]{cleveref}

\crefname{listing}{Listing}{Listings}
\Crefname{listing}{Listing}{Listings}
\crefname{lstlisting}{Listing}{Listings}
\Crefname{lstlisting}{Listing}{Listings}

\usepackage{lipsum}

% For demonstration purposes only
% These packages can be removed when all examples have been deleted
\usepackage[math]{blindtext}
\usepackage{mwe}
\usepackage[realmainfile]{currfile}
\usepackage{tcolorbox}
\tcbuselibrary{listings}

%introduce \powerset - hint by http://matheplanet.com/matheplanet/nuke/html/viewtopic.php?topic=136492&post_id=997377
\DeclareFontFamily{U}{MnSymbolC}{}
\DeclareSymbolFont{MnSyC}{U}{MnSymbolC}{m}{n}
\DeclareFontShape{U}{MnSymbolC}{m}{n}{
  <-6>    MnSymbolC5
  <6-7>   MnSymbolC6
  <7-8>   MnSymbolC7
  <8-9>   MnSymbolC8
  <9-10>  MnSymbolC9
  <10-12> MnSymbolC10
  <12->   MnSymbolC12%
}{}
\DeclareMathSymbol{\powerset}{\mathord}{MnSyC}{180}

\usepackage[
  translate=babel,
  abbreviations,         % create "abbreviations" glossary
  nomain,                % don't create "main" glossary
  stylemods=longbooktabs % do the adjustments for the longbooktabs styles
]{glossaries-extra}
\setglossarystyle{long3col-booktabs}

% Hint by https://tex.stackexchange.com/a/463188/9075
% \usepackage{glossary-longextra}

% Following is required if the abbreviation list should be sorted automatically (\printglossary / \printglossaries)
% Not required, if we printed the entries in-order (using \printunsrtglossaries)
% Required to have the German chapter name % Source: https://tex.stackexchange.com/a/426392/9075
\makeglossaries

\input{abbreviations}

% Allows for defining commands that don't eat spaces.
\usepackage{xspace}
% Adds compatibility to \xspace und \enquote
\makeatletter
\xspaceaddexceptions{\grqq \grq \csq@qclose@i \} }
\makeatother

\newcommand{\eg}{e.g.,\ }
\newcommand{\ie}{i.e.,\ }

% Enable hyphenation at other places as the dash.
% Example: applicaiton\hydash specific
\makeatletter
\newcommand{\hydash}{\penalty\@M-\hskip\z@skip}
% Definition of "= taken from http://mirror.ctan.org/macros/latex/contrib/babel-contrib/german/ngermanb.dtx
\makeatother

% Add manual adapted hyphenation of English words
% See https://ctan.org/pkg/hyphenex and https://tex.stackexchange.com/a/22892/9075 for details
\input{ushyphex}

% correct bad hyphenation here
\hyphenation{
  op-tical net-works semi-conduc-tor
  % May not be hypphenated
  AROMA TOSCA BPMN OASIS OMG DMTF IT DevOps
}

\input{commands}

% Package URL: https://ctan.org/pkg/scientific-thesis-cover
\usepackage[
  title={Insights from a very large EEG+Eyetracking dataset},
  author={Okan Mazlum},
  type=master,
  institute=vis, % or other institute names - or just a plain string using {Demo\\Demo...}
  course={Informatik},
  examiner={Jun.\ -Prof.\ Dr.\ Benedikt Ehinger},  
  supervisor={Dr.\ Jevri Hanna},
  startdate={October 13, 2025},
  enddate={April 13, 2026}
]{scientific-thesis-cover}


\ifpdftex
  % Enable copy and paste of text from the PDF
  % Only required for pdflatex. It "just works" in the case of lualatex.
  % Alternative: cmap or mmap package
  % mmap enables mathematical symbols, but does not work with the newtx font set
  % See: https://tex.stackexchange.com/a/64457/9075
  % Other solutions outlined at http://goemonx.blogspot.de/2012/01/pdflatex-ligaturen-und-copynpaste.html and http://tex.stackexchange.com/questions/4397/make-ligatures-in-linux-libertine-copyable-and-searchable
  % Trouble shooting outlined at https://tex.stackexchange.com/a/100618/9075
  %
  % According to https://tex.stackexchange.com/q/451235/9075 this is the way to go
  \input{glyphtounicode}
  \pdfgentounicode=1
\fi
% DM: line-breaking-description env vom daniel w.

% credit goes to daniel w. :-)
%% --- Descriptions with line breaks in labels ---------------------------------
\usepackage{calc}

\newcommand*\Descriptionlabel[1]{%
  \raisebox{0pt}[1ex][0pt]{
    \makebox[\labelwidth][1]{
      \parbox[t]{\labelwidth}{
        \hspace{0pt}\textbf{#1:}}}}
}

\newcommand*\Descriptionlabelx[1]{%
  \parbox[t]{\textwidth}{
    \textbf{#1}\\\mbox{}}
}

\newenvironment{Description}{
  \begin{list}{}{
      \let\makelabel\Descriptionlabelx
      \setlength\labelwidth{1em}
      \setlength\leftmargin{\labelwidth+\labelsep}
    }
    }
    {
  \end{list}
}

% globally change line spacing of lists
% paralist has suspended development since 10 years.
% enumitem has been updated 2011-09-28
\usepackage[inline]{enumitem}
\setlist{partopsep=0pt,itemsep=1pt}

%------------------------------------------------------------------------
% fquote Fancy Quotation environment
% supports empty/optional author

% Use \sloppy to make right-margin easier?
% Set picture units to be relative to font size (em)?
% Use begingroup to rest units afterwards?

\usepackage{xifthen}% provides \isempty test
\definecolor{quotemark}{gray}{0.7}

%fquote environment with author as optional parameter
%usage: \begin{fquote}quote\end{fquote} or \begin{fquote}[Author]quote\end{fquote}
\newenvironment{fquote}[1][]{%
  \newcommand{\fqauthor}{\relax}
  \ifthenelse{\isempty{#1}}
  {}% do nothing
  {\renewcommand{\fqauthor}{\hfill\textsc{--- #1}}}
  \vspace{1em}
  \begin{list}{}{%
      \setlength{\leftmargin}{0.2\textwidth}
      \setlength{\rightmargin}{0.2\textwidth}
    }
    \item[]%
          \begin{picture}(0,0)(0,0)
            \put(-15,-5){\makebox(0,0){%
                \scalebox{4.5}{\textcolor{quotemark}{\bfseries``}}}%
            }
          \end{picture}\em\ignorespaces%
          }{%
          \newline%
          \makebox[0pt][l]{\hspace{0.6\textwidth}%
            \begin{picture}(0,0)(0,0)
              \put(15,10){\makebox(0,0){%
                  \scalebox{4.5}{\textcolor{quotemark}{\rmfamily\bfseries''}}}%
              }
            \end{picture}}%
          \fqauthor
  \end{list}
}

%German fquote
%  1 parameter for the author's name, may be empty ("{}")
%  guaranteed German quotes (works with lualatex and babel package)
%  usage: \begin{gfquote}{Author}quote\end{gfquote}
\newenvironment{gfquote}[1]{%
  \newcommand{\fqauthor}{\relax}
  \ifthenelse{\isempty{#1}}
  {}% do nothing
  {\renewcommand{\fqauthor}{\hfill\textsc{\textemdash #1}}}
  \vspace{1em}
  \begin{list}{}{%
      \setlength{\leftmargin}{0.2\textwidth}
      \setlength{\rightmargin}{0.2\textwidth}
    }
    \item[]%
          \begin{picture}(0,0)(0,0)
            \put(-15,-5){\makebox(0,0){%
                \scalebox{4.5}{\textcolor{quotemark}{\bfseries \glqq}}}%
            }
          \end{picture}\em\ignorespaces%
          }{%
          \newline%
          \makebox[0pt][l]{\hspace{0.6\textwidth}%
            \begin{picture}(0,0)(0,0)
              \put(15,10){\makebox(0,0){%
                  \scalebox{4.5}{\textcolor{quotemark}{\rmfamily\bfseries \grqq}}}%
              }
            \end{picture}}%
          \fqauthor
  \end{list}
}

% fix incompatibilities between KOMA and other packages, mainly float.
% should be loaded at the very end - see http://tex.stackexchange.com/a/156256/9075
\usepackage{scrhack}


\begin{document}
\raggedbottom
\pagenumbering{arabic}
\Titelblatt

\pagestyle{plain.scrheadings}
\renewcommand*{\chapterpagestyle}{plain.scrheadings}

% abstract
% Same style as table of contents
\section*{Abstract}
\emph{sample abstract
  \footnote{sample footnote}}

\emph{Introduction.}


\microtypesetup{protrusion=false}

% In case you have trouble with headings reaching into the page numbers, enable the following three lines.
% Hint by http://golatex.de/inhaltsverzeichnis-schreibt-ueber-rand-t3106.html
%
%\makeatletter
%\renewcommand{\@pnumwidth}{2em}
%\makeatother
%
% In case of a strange break in the table of contents,
% a page break can be inserted by issuing the following command at the "right" place in the main text:
%  \addtocontents{toc}{\protect\newpage}
\tableofcontents

\listoffigures

\listoftables

% We use lstlisting environments with caption paramters.
% Thus, we need that command.
% Alternative: \listof{Listing}{List of Listings}
\lstlistoflistings

% mittels \newfloat wurde die Algorithmus-Gleitumgebung definiert.
% Mit folgendem Befehl werden alle floats dieses Typs ausgegeben
%\listof{Algorithmus}{List of Algorithms}
%\listofalgorithms %Ist nur für Algorithmen, die mittels \begin{algorithm} umschlossen werden, nötig

% Abbreviations / Acronyms
%\printglossary[type=\acronymtype,title={Abbreviations}]
% \printglossaries
% \printnoidxglossaries
% \printunsrtglossaries cannot be used, because then no indexing happens; source: https://tex.stackexchange.com/a/287128/9075

\microtypesetup{protrusion=true}

% Headline and footline
\renewcommand*{\chapterpagestyle}{scrplain}
\pagestyle{scrheadings}

%%% ===============================================================================
\chapter{Introduction}\label{sec:introduction}
%%% ===============================================================================

\ldots \mynote{proper intro}

\section{Datasets}\label{ssec:datasets}
\ldots \ldots \mynote{proper explaination of what the purposes and differences of the HBN and HBN-EEG datasets are}
\subsection{Paradigms}\label{sssec:paradigms}
%%% itemize
\begin{itemize}
  \item \ldots
  \item \textbf{Symbol Search:} \mynote{quick explanation of symbol search paradigm. it can be seen how the subject moves their gaze horizontally line by line. the image captures only the first page, after pressing a button, this gaze pattern repeats (starting at the top left). i wanted to include this image because i think it nicely visualizes the et data that will be analyzed (very horizontal-heavy eye movement, shows types of saccades that are performed, new stimulus onset + large saccade with every new page)}
\end{itemize}
\begin{figure}[htb]
  \centering
  \includegraphics[width=0.5\textwidth]{gaze_history.png}
  \caption{Gaze history visualization \mynote{Clean up this image and maybe find a better subject}}
  \label{fig:gaze_history}
\end{figure}


\chapter{Related Work}\label{sec:relatedwork}

\ldots

\chapter{Data and Methods}\label{sec:datanmethods}

\section{Data acquisition and sources}

\mynote{The part on datasets in the Introduction covers the datasets on a basic level, here some more in-depth technical description is needed to aid in understanding the next sections. this also needs more detail}

\subsubsection{HBN-EEG dataset}
The dataset was retrieved from NeMAR \cite{nemar-hbn-eeg} and consists of 11 \texttt{.zip} archives totalling 1.7 TiB, each containing curated EEG data in BIDS format.

\subsubsection{HBN dataset}
The original HBN dataset is hosted on the public fcp-indi S3 bucket \cite{bucket}. The folder \texttt{data/Archives/HBN/EEG/} contains 4576 \texttt{.tar.gz} subject archives totalling 5.6 TiB. Each subject archive holds at most 3 Folders, \texttt{Behavioral}, \texttt{EEG} and \texttt{Eyetracking} \mynote{double check logs whether subjects exist where not all 3 are present}.
\begin{itemize}
  \item \textbf{Behavioral:} Contains phenotypic data about the subject in at most two different formats, \texttt{.csv} and \texttt{.mat}.
  \item \textbf{EEG:} Contains preprocessed \footnote{The exact preprocessing steps that were applied are unknown and the preprocessed data is not meant to be used \url{https://www.nitrc.org/forum/forum.php?forum_id=10003&thread_id=15454} } and raw EEG data. Since the HBN-EEG dataset will be used over the original HBN dataset for EEG data, this folder is of no further interest.
  \item \textbf{Eyetracking:} Contains at most 4 subfolders: \texttt{idf}, \texttt{txt}, \texttt{mat} and \texttt{tsv} \mynote{actually check whether subjects exist where all 4 are present}. Each folder represents the same eyetracking data in a different format. \mynote{go into a little more detail here, specifically that tsv ET does not contain necessary events to process it, that the idf folder is always empty, and that mat requires conversion to .txt}
\end{itemize}

Based on the release documentation \cite{hbn-releases}, updates between releases mostly concern data availability and dataset curation (e.g., additions of participants, corrections to phenotypic tables or metadata). In particular, there are no changes that would require release-specific treatment of the EEG data. For this reason, EEG data from the full HBN-EEG collection were merged together to form a single integrated BIDS dataset. This enables a single, consistent preprocessing and analyis pipeline across all subjects. This merged dataset was then enriched with the associated eye-tracking and phenotypic data for each subject from the original HBN archives. The integration of this type of data is not yet standardized in BIDS, and this step will cause BIDS validators \cite{bids-validator} to consider the dataset as invalid. The tools used in this study (mne-bids-pipeline and Unfold.jl) remained compatible with this developmental format, simply ignoring the extra files that are not of relevance. The following sections cover how this merged dataset was created.

\subsection{Download}

Custom helper scripts were used to download and unpack all required source archives in a restartable manner. This was done to improve the reproducibility of the study compared to manual downloading.
Downloads were performed with parallel workers and to ensure integrity of the files, incomplete transfers were written to a staging directory and only moved to the final location after the expected file size was reached. This approach was taken both for the retrieval from the fcp-indi S3 bucket and for the web download from NeMAR.
After download, all archives were extracted using dedicated unpacking scripts making use of the same paradigms to avoid partial extractions and make use of parallelization.

\subsection{Merging}

Finally, further scripts make use of the downloaded files to deterministically create the final merged BIDS dataset. For files that are required as-is in the new merged dataset, e.g. EEG and eye-tracking recordings, the script creates symlinks to the downloaded files and renames them to be BIDS conform. Other files like the \texttt{participants.tsv} or other metadata files for the merged dataset have to be generated by merging metadata files from the separate HBN-EEG releases (i.e. concatenating all \texttt{participants.tsv} or some summary tables together into one file). To support traceability, some files were also augmented with their source release numbers. For files that were contained in all releases with only minor differences irrelevant to the merged result, the file from the first release was chosen.

Because metadata conventions and file naming practices vary slightly across releases, some harmonization steps were applied during dataset construction. These include:

\begin{enumerate}
  \item Standardization of BIDS file names (subject identifiers, task labels, run indices, and separators)
  \item Completion and canonicalization of \texttt{$\ast$\_channels.tsv} contents by adding required columns and enforcing consistent channel types and units
  \item Normalization of tabular files by harmonizing delimiters, column names, and column order
  \item Targeted handling of a small number of subjects (< 20) with irregularities beyond these standard cases (e.g. by hardcoded renaming, conversion, or exclusion).
\end{enumerate}

Throughout the merging process, the scripts validated expected directory structure and file contents, logging and/or terminating on any unexpected finds.

\subsection{Synchronization with eye-tracking data}

\mynote{reformulate and majorly elaborate on these paragraphs, depending on what will be written in the introduction}
To synchronize EEG data with eye-tracking data, a fork of the mne-bids-pipeline was used, which is currently being developed for this exact purpose.
It works by lining up shared events in the EEG and eye-tracking recording. The outputted \texttt{.fif} files contain eye-tracking samples and channels alongside the usual EEG data.

Due to the number of shared events being the only grounds for synchronization accuracy, the movie watching paradigms in the dataset were unsuitable for this study, containing only 2 shared events per recording. This only left the symbol search paradigm \ref{sssec:paradigms} as a candidate to analyze \mynote{depending on how quality scoring turns out, some movie recordings will likely still be used if deemed to have good synchronization despite small shared event count}.

\mynote{@jevri if i have a movie watching run, and all quality metrics look good, is it fine to still consider it for analysis even though it only has two shared events? or should i blanket ban all recordings with shared events under some treshold? i feel like if the xcorr curve looks good, it still might be worth considering}

\subsection{Parsing of eye-tracking data}

Eye-tracking data in the HBN dataset were recorded using the iView-X Red-m by SensoMotoric Instruments \cite{hbn}. These recordings usually come in a proprietary \texttt{.idf} format, but were converted to \texttt{.txt} and \texttt{.tsv} for almost all subjects \mynote{give exact number. maybe mention that despite the converter being obsolete/abandonware, there seem to exist alternatives for conversion. however no .idf files remain in the dataset so this is irrelevant}.

An eye-tracking recording has two files associated with it. 

\begin{itemize}
  \item Samples file (\texttt{sub-<id>\_task-symbolSearch\_et.txt}): \mynote{briefly describe structure of this file} \mynote{essentially, the file starts with some headers containing a few useful pieces of information, followed by a tsv (with slight quirks). this table lists all ET samples as well as all user events with their respective timestamps}
  \item Events file (\texttt{sub-<id>\_task-symbolSearch\_et\_Events.txt}): \mynote{briefly describe structure of this file} \mynote{again, it's some headers followed by a quasi-tsv. this time the tsv contains all user events (these were confirmed to be fully identical to the user events in the samples file) and saccade, blink, and fixation events. these events include fixation duration, position, etc...}
\end{itemize}

Subjects may only have an Events file without Samples file \mynote{how many subjects have this property?} in which case an analysis is still possible to some degree, as detailed gaze positions are not required for all analyses performed in this study.
Similarily, for subjects that have a Samples file without an Events file \mynote{how many subjects have this property?}, saccade and fixation information can be recreated from eye-tracking information \mynote{this is not currently done}.
Only subjects with both files missing \mynote{how many subjects?} are skipped.

The used mne-bids-pipeline fork exclusively supported eye-tracking files in eyelink format (\texttt{.asc}, \texttt{.edf}), and had to be extended with the functionality to parse converted \texttt{.idt} files. For later curation of select subjects, numerous metrics were additionally collected during the parsing and synchronization process \mynote{this step has to be done because manual curation is not feasible in a dataset of this size}. As this step is ultimately very specific to this study, a second variant of the modified mne-bids-pipeline was created that exclusively focused on adding support for \texttt{.idt} files.

\mynote{quick summary of \texttt{read\_raw\_iview}. formulate this properly} \\
\mynote{[if samples file is present]} \\
\mynote{read sample rate from header} \\
\mynote{locate start of tsv, special treatment for time and type cols} \\
\mynote{remove channels that only have nan values} \\
\mynote{count broken samples, i.e. sample has almost all gaze data == 0 (maybe omit as it's mentioned later in metrics)} \\
\mynote{parse user events (type == \"MSG\") into mne Annotations} \\
\mynote{estimate actual sampling frequency from times channel, as otherwise some annotations may fall outside of the raw time range} \\
\mynote{add everything to Raw object} \\
\\
\mynote{if events file is present} \\
\mynote{parse user events (type == \"UserEvent\") into mne Annotations} \\
\mynote{parse Fixation, Blink, Saccade events into mne Annotations. Make use of \"extras\" attributes that can be added to Annotations since MNE-Python v1.10.0 to add the auxiliary info like fixation position, saccade speed, ...}

\ldots \ldots \\

\mynote{misc hurdles to maybe mention} \\
\mynote{mne internal function \texttt{\_combine\_annotations} (concatenating a tuple of annotations) seems to not support \"extras\". when attempting, the resulting combined annotation object is missing all \"extras\" info. to fix this, the internal function was adjusted} \\
\mynote{some ET \texttt{.txt} files contained special symbols that crashed mne when trying to name the channels, so they had to be renamed} \\
\mynote{adjusted prefixing events with \texttt{ET\_}, which previously caused event messages to be tructuated. look at code for this again, it might have slightly modified functionality} \\
\mynote{when texttt{first\_samp} differs between streams, shift back EEG annotations so they are synced. without this step, some annotations may again fall out of range}
\mynote{added assert that at least 2 sync events are needed, as regression crashes otherwise}
\\

\subsection{Capturing of synchronization quality metrics}

Throughout the parsing and synchronization process, metrics are collected to assess the quality of the input EEG and Eye-tracking files, as well as the resulting synchronized file. These metrics include meta properties (sample counts, sampling rates, channel counts), basic checks (number of invalid samples, average saccade, fixation and blink properties) and previously built-in synchronization quality feedback (regression slope/intercept, residual timing error with counts within ±1 and ±4 EEG samples, cross-correlation between HEOG and screen gaze position).
For a full list of metrics, see the table in the Appendix \mynote{todo}.

Once finished, the script writes all metrics as a as \texttt{$\ast$\_metrics.json} into the derivatives folder next to the standard mne-bids-pipeline output. Additionally, an \texttt{$\ast$\_xcorr-artifact.npz} file stores part of the cross-correlation plot.
A helper script combines all \texttt{.json} files to generate an overview over every recording as \texttt{.xlsx} \mynote{maybe update to store as .csv}

\subsubsection{Cross correlation plot}

The cross-correlation is computed between the horizontal EOG (HEOG) signal and the eye-tracker’s horizontal gaze position (``L POR X [px]'' and ``R POR X [px]'', averaged if both are present). If both streams are correctly time-aligned, EOG potential and gaze position movement should co-occur and produce a strong correlation at (or very near) zero lag.

\mynote{very unsure about this section. jevri said these plots should look like sharp normal distributions, not like pyramids}

An exemplary cross correlation plot between gaze position and EOG can be found in the documentation of the EYE-EEG toolbox. As can be seen, the cross correlation peaks at zero lag with a parabola-like shape at the tip, and with symmetric linear lobes.

\mynote{when i was identifying EOG channels, jevri said it should have step/box-like shapes in the signal. also, the gaze position signal (L POR X) also has box-like shapes. it would make sense if xcorr between two step functions gives a triangle with linear lobes. i need actual pictures of the data if i want to make this claim though}

\mynote{@jevri: do you have an idea why this xcorr curve could look so different from the one outputted by the mne-bids-pipeline fork? I looked at the code and this should also just be the xcorr between EOG and gaze position (not gaze velocity or something like that)}

\begin{figure}[htb]
  \centering
  \includegraphics[width=0.8\textwidth]{syncquality_tobii.jpg}
  \caption{Cross-correlation plot from EYE-EEG \cite{eyeeeg}}
  \label{fig:syncquality_tobii}
\end{figure}

\mynote{maybe mention band pass filtering done in the fork}

To automatically assess the desirability of the cross correlation curve, some dedicated measurements are made, including:
\begin{itemize}
  \item \textbf{Peak position and height:} The peak position should be close to 0 lag. The height strongly depends on the subject, but it should be the highest peak in the entire plot. Additionally, measurements of the second highest peak in the plot are made, so it can be determined how prominent/distinct the center spike is. If the second highest peak of the cross correlation curve is very far from 0 lag and almost as high, it may be an indicator that the center peak is just coincidence \mynote{reformulate, very bad paragraph} 
  \item \textbf{Similarity to `template curve':} As seen in \cref{fig:syncquality_tobii}, an optimal cross correlation curve will have linear lobes downwards from its peak. Using the value of the plot at zero lag and computed endpoints for the lobes on either side, a triangular `template curve' is constructed, that is expected to have high similarity with the cross correlation curve iff the signals are well aligned \mynote{reformulate and also elaborate here. specifically how exactly are the endpoints for the lobes chosen: usually where the xcorr plot crosses y=0, but if this does not happen, it attempts to find prominent valleys instead}. \mynote{the similarity with the template curve is calculated in different ways like cosine similarity and kullback leibler divergence}
  \item \mynote{lobe steepness of the template curve: steeper is better. also symmetric (i.e. similar steepnesses for both lobes) is probably also better?}
  \item \mynote{kurtosis of the cross correlation curve. todo}
  \item \mynote{how many peaks are around the center of the plot? to see if it is 1 prominent peak or many peaks}
\end{itemize}

\mynote{could it make sense to test the xcorr curve with different offsets and see how high the highest peak is with these random offsets? if the peaks are just as high as when synchronized, the xcorr curve is probably worthless for this subject}


\subsection{Evaluation of quality metrics}

\mynote{add histograms of select metrics where interesting} \\
\mynote{create scatterplot of median\_abs\_sync\_error\_ms vs snr. could be interesting to see how strong the correlation is, and whether there are any clusters} \\
\mynote{research if there is an established treshold for xcorr value in literate} \\
\mynote{perhaps the only `objective' metric of synchronization quality is the median\_abs\_sync\_error\_ms. if it's off by more than 2 ET samples on average, synchronization is probably off} \\
\mynote{multiple things have already been attempted here, but probably nothing that will remain in its current form, so not documenting it here yet} \\


\section{Preprocessing}

\mynote{entire section is just notes for now}

\begin{verbatim}

# needs to be set to ignore analysis and only do preprocessing
task_is_rest = True
# these need to be set also
# see https://github.com/mne-tools/mne-bids-pipeline/issues/1020
rest_epochs_duration = 5
rest_epochs_overlap = 0

# these are forced by ICALABEL
l_freq: float | None = 1
h_freq: float | None = 100
eeg_reference = "average"

# data was recorded in the US
notch_freq = 60

# size of the dataset requires icalabel
ica_algorithm = "picard-extended_infomax"
ica_use_icalabel = True
ica_reject = "autoreject_local" 

# no epoch rejection, we need continuous data
reject = None 

# enable eye-tracking synchronization
sync_eyelink = True
sync_eye = True

# cover all symbol search trigger events
sync_eventtype_regex     = r"(?:trialResponse|newPage)" #r"trialResponse"
sync_eventtype_regex_et  = r"# Message: (?:14|20)" #r"# Message: 14"

# HBN setup paper lists the following electrodes as being EOG electrodes:
# 8, 14, 17, 21, 25, 125, 126, 127, 128
# looking at the 3d montage, this is the only
# set of these electrodes that makes sense here
eeg_bipolar_channels = {
    "HEOG": ("E8", "E25"),   # left vs right outer canthus
    "VEOG": ("E21", "E17"),  # nasion vs left inner canthus
}

eog_channels = ["HEOG", "VEOG"]
sync_heog_ch = "HEOG"
sync_et_ch = ("L POR X [px]", "R POR X [px]")
montage = mne.channels.make_standard_montage("GSN-HydroCel-128")
eeg_template_montage = montage
drop_channels = ["Cz"]

\end{verbatim}










%%% ===============================================================================
\chapter{TODO}\label{sec:conclusion}
%%% ===============================================================================

- use consistent tense when writing (currently switching between past and present) \\
- properly cite everything \\


%%% ===============================================================================
%%% Bibliography
%%% ===============================================================================


\printbibliography

% Enfore empty line after bibliography
\ \\
%
\noindent
All links were last followed on October 5, 2020.

%%% ===============================================================================

%\IfDefined{printindex}{\printindex}
%\IfDefined{printnomenclature}{\printnomenclature}

\clearpage
\appendix
% 'Anhang' ins Inhaltsverzeichnis
%\phantomsection
%\addcontentsline{toc}{chapter}{Appendix}
\addcontentsline{toc}{part}{Appendix}

%%% ===============================================================================
\chapter{My first appendix}\label{sec:appendix1}
%%% ===============================================================================

%\lipsum[1]
Sample Appendix

\pagestyle{empty}
\renewcommand*{\chapterpagestyle}{empty}
\Versicherung
\end{document}
